\documentclass{article}

\usepackage{graphicx}
\usepackage{lipsum}
\usepackage{caption}
\usepackage{hyperref} % Para adicionar links
\usepackage{enumitem} % Para personalizar listas

% Adiciona o pacote para usar Helvetica como fonte sem serifa
\renewcommand{\rmdefault}{phv} % Usa Helvetica para o texto principal
\renewcommand{\sfdefault}{phv} % Usa Helvetica para títulos, seções, etc.

\begin{document}

\begin{flushleft}
    \textbf{Press Release}
\end{flushleft}

\section*{\Huge Indian physicists suggest the next sunspot cycle will likely peak in January 2024}

\begin{flushleft}
    7 December 2023
\end{flushleft}
\begin{flushleft}
    Written by Gustavo Pires Bertaco
\end{flushleft}

\begin{figure}[h]
    \centering
    \includegraphics[width=1\linewidth]{image.png}
    \captionsetup{justification=justified} 
    \caption*{\textbf{ Using 47 years data from the World Data Center SILSO and Wilcox Solar Observatory, researchers at the Indian Center of Excellence in Space Sciences discover a relationship between the decay rate of the Sun's magnetic dipole and the growth rate of the next 11-year solar cycle of sunspots, which could be a new precursor to help with this long-term prediction that is crucial for forecasting solar storms and even in the planning of future space missions. Credit: HMI/SDO/NASA}}
\end{figure}

Our Sun, a hot star ionized gas known as plasma orchestrates an intricate dance of magnetic activity every 11 years, punctuated by the emergence and dissipation of sunspots. These darkened patches on the Sun's surface signify the culmination of a magnetic field process, resulting in regions temporarily cooler and darker than their surroundings. Sunspots are comparable to the size of Earth, and have diameters ranging from 16 km to 160,000 km.\\

The crux of our discovery lies in the correlation between the decline of the Sun's magnetic dipole moment and the subsequent surge in sunspot activity. This link unveils a connection with the Babcock-Leighton mechanism, shedding light on the fundamental dynamics of solar polar field generation.\\

Observations indicate that the reversal of the dipole moment's polarity heralds the imminent peak of the sunspot cycle, occurring approximately a year later. Given the last observed polarity reversal in July 2022, our analysis predicts the impending climax of Solar Cycle 25, likely to reach its zenith in January 2024, with a plausible range spanning from July 2023 to September 2024.\\

This forecast, grounded in the interplay of solar magnetic forces, not only aligns with empirical data but converges with the predictions of Bhowmik \& Nandy in 2018. \\

\textit{“The existence of such a strong correlation, in fact, enables one to forecast the timing of a sunspot cycle’s peak once the amplitude of that cycle is independently anticipated. For example, we show that the ongoing sunspot cycle is likely to peak during January 2024 with an estimated amplitude that matches the physical model-based prediction of Bhowmik.”}\\

 Such synchronization bodes well for the field of solar cycle predictions, offering a promising alternative method that could enhance our ability to anticipate  forecasting the timing of the peak of solar cycles, where intense activity and a most frequent space weather disturbances are more expected.

\section*{\LARGE More Information}

This research was presented in a paper titled “Discovery of a relation between the decay rate of the Sun’s magnetic dipole and the growth rate of the following sunspot cycle: a new precursor for solar cycle prediction" to appear in \textit{Monthly Notices of the Royal Astronomical Society:Letters, Volume 528, Issue 1, February 2024, Pages L27–L32, \href{https://doi.org/10.1093/mnrasl/slad122}{https://doi.org/10.1093/mnrasl/slad122}}.\\



The European Southern Observatory (ESO) enables scientists worldwide to discover the secrets of the Universe for the benefit of all. We design, build and operate world-class observatories on the ground — which astronomers use to tackle exciting questions and spread the fascination of astronomy — and promote international collaboration for astronomy. Established as an intergovernmental organisation in 1962, today ESO is supported by 16 Member States (Austria, Belgium, the Czech Republic, Denmark, France, Finland, Germany, Ireland, Italy, the Netherlands, Poland, Portugal, Spain, Sweden, Switzerland and the United Kingdom), along with the host state of Chile and with Australia as a Strategic Partner. ESO’s headquarters and its visitor centre and planetarium, the ESO Supernova, are located close to Munich in Germany, while the Chilean Atacama Desert, a marvellous place with unique conditions to observe the sky, hosts our telescopes. ESO operates three observing sites: La Silla, Paranal and Chajnantor. At Paranal, ESO operates the Very Large Telescope and its Very Large Telescope Interferometer, as well as survey telescopes such as VISTA. Also at Paranal ESO will host and operate the Cherenkov Telescope Array South, the world’s largest and most sensitive gamma-ray observatory. Together with international partners, ESO operates ALMA on Chajnantor, a facility that observes the skies in the millimetre and submillimetre range. At Cerro Armazones, near Paranal, we are building “the world’s biggest eye on the sky” — ESO’s Extremely Large Telescope. From our offices in Santiago, Chile we support our operations in the country and engage with Chilean partners and society. 


\section*{\LARGE Links}

\begin{itemize}[label=$\bullet$] % Personaliza a lista com bolinhas
    \item \textbf{\href{https://doi.org/10.1093/mnrasl/slad122}{Research paper}} 
    %\item \textbf{Photos of ALMA} -- link
\end{itemize}

\section*{\LARGE Contacts} 

Dibyendu Nandi\\
Center of Excellence in Space Sciences India, Indian Institute of Science Education and Research Kolkata.\\
Department of Physical Sciences, Indian Institute of Science Education and Research Kolkata.\\
Mohanpur 741246, West Bengal-India\\
Tel: +91 33 61360045\\
E-mail: dnandi@iiserkol.ac.in


\begin{flushleft}
    \rule{6cm}{0.5pt}\\
    {\footnotesize{This press release was written in \LaTeX{} based on ESO press releases by BERTACO, G. P. .}}
\end{flushleft}

\end{document}
